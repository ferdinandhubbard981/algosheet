
\documentclass{article}
\usepackage{graphicx}
\graphicspath{ {./images/} }
\usepackage[landscape]{geometry}
\usepackage{url}
\usepackage{multicol}
\usepackage{amsmath}
\usepackage{esint}
\usepackage{amsfonts}
\usepackage{tikz}
\usetikzlibrary{decorations.pathmorphing}
\usepackage{amsmath,amssymb}

\usepackage{colortbl}
\usepackage{xcolor}
\usepackage{mathtools}
\usepackage{amsmath,amssymb}
\usepackage{enumitem}
\makeatletter

\newcommand*\bigcdot{\mathpalette\bigcdot@{.5}}
\newcommand*\bigcdot@[2]{\mathbin{\vcenter{\hbox{\scalebox{#2}{$\m@th#1\bullet$}}}}}
\makeatother

\title{Algorithms II Cheat Sheet}
\usepackage[brazilian]{babel}
\usepackage[utf8]{inputenc}

\advance\topmargin-.8in
\advance\textheight3in
\advance\textwidth3in
\advance\oddsidemargin-1.5in
\advance\evensidemargin-1.5in
\parindent0pt
\parskip2pt
\newcommand{\hr}{\centerline{\rule{3.5in}{1pt}}}
%\colorbox[HTML]{e4e4e4}{\makebox[\textwidth-2\fboxsep][l]{texto}
\begin{document}

\begin{center}{\huge{\textbf{Algorithms II Cheat Sheet}}}\\
\end{center}
\begin{multicols*}{3}

\tikzstyle{mybox} = [draw=black, fill=white, very thick,
    rectangle, rounded corners, inner sep=10pt, inner ysep=10pt]
\tikzstyle{fancytitle} =[fill=black, text=white, font=\bfseries]

%------------ tips ---------------
\begin{tikzpicture}
\node [mybox] (box){%
    \begin{minipage}{0.3\textwidth}
    Apply an algorithm you know in a clever way, don't write a new algorithm.
    \end{minipage}
};
%------------ tips Header ---------------------
\node[fancytitle, right=10pt] at (box.north west) {Tips};
\end{tikzpicture}


%------------ Notation ---------------
\begin{tikzpicture}
\node [mybox] (box){%
    \begin{minipage}{0.3\textwidth}
    $A \in [10] \equiv A \in [1..10]$ \\
    \{a, b c\} is a set of vertices \\
    G\{a, b, c\} is a graph
    \end{minipage}
};
%------------ Notation Header ---------------------
\node[fancytitle, right=10pt] at (box.north west) {Notation};
\end{tikzpicture}
    
%------------ Big O ---------------
\begin{tikzpicture}
\node [mybox] (box){%
    \begin{minipage}{0.3\textwidth}
        \includegraphics[width=8cm]{big-o-definition.png}
        \includegraphics[width=8cm]{big-o-formal.png}
    \end{minipage}
};
%------------ Big O Header ---------------------
\node[fancytitle, right=10pt] at (box.north west) {Big O};
\end{tikzpicture}

%------------ Interval Scheduling ---------------
\begin{tikzpicture}
\node [mybox] (box){%
    \begin{minipage}{0.3\textwidth}
    \includegraphics[width = 8cm]{req-comp-def.png}
    \hline
    \includegraphics[width = 8cm]{interval-scheduling-problem.png}
    \hline
    \includegraphics[width=8cm]{greedy-schedule-pseudo.png}
    \textbf{Complexity:} \\
    Step 2 takes O(n log n) \\
    Steps 3–6 all take O(1) time and are executed at most n times. \\
    $\therefore total running time = O(n log n) + O(n) · O(1) = O(n log n).$
    \end{minipage}
};
%------------ Interval Scheduling Header ---------------------
\node[fancytitle, right=10pt] at (box.north west) {Interval Scheduling};
\end{tikzpicture}


%------------ Interval Scheduling 2 ---------------
\begin{tikzpicture}
\node [mybox] (box){%
    \begin{minipage}{0.3\textwidth}
    
    \textbf{Formal GreedySchedule} \\
    $A^+$ := argmin $\{f : (s, f ) \in R, A \cup \{(s, f )\}$ is compatible\} for all $A \subseteq R$, \\
    $A_0 := \emptyset,$ \hspace{1cm} $A_{i+1} := A_i \cup \{A_i^+\}$ \\
    t := max\{i: $A_i$ is defined\}\\
    \textbf{Interval Scheduling Proofs} \\
    \textbf{Lemma}: Greedy Schedule always outupts $A_t$ \\
    \textbf{Proof}: By induction form the following loop invariant.
    At the start of the i'th iteration of 4-7: \\
    \vspace{-5mm}
    \setlist{nolistsep}
    \begin{itemize} [noitemsep]
        \item A is equal to $A_t \cap \{(s_1, f1), ..., (s_{i-1}, f_{i-1})\}$
        \item lastf is equal to the latest finish time of any request in A (or 0 if A = [])
    \end{itemize}
    \hline
    \textbf{Lemma}: $A_t$ is a compatible set \\
    \textbf{Proof}: Instant by induction; $A_0$ is compatible, and if $A_i$ is compatible then so is $A_{i+1} = A_i \cup {A^+}$ by the definition of $A_i^+$
    
    \hline
    \textbf{Lemma}: $A_t$ is a maximum compatible subset of the Array $R$ (look in pseudocode) \\
    \textbf{Proof}:\\
    Base case for i = 1: $A_0^+$ is the fastest finishing request in $R$ by definition\\
    Inductive step: Suppose $A_i$ finishes faster than $B_i$. \\
    Let $B_i^+$ be the (i+1)'st fastetst-finishing element of B.
    Since $A_i$ finishes faster than $B_i$, $A_i \cup \{B_i^+\}$ is compatible. Hence by definition, $A_i^+$ exists and finishes no later than $B_i^+$
    \hline
    \textbf{Theorem}: GreedySchedule outputs $A_t$, which is a maximum compatible set. \\
    \textbf{Proof}: putting all of the above proofs together, we prove the theorem.
    \end{minipage}
};
%------------ Interval Scheduling 2 Header ---------------------
\node[fancytitle, right=10pt] at (box.north west) {Interval Scheduling};
\end{tikzpicture}

%------------ Graph Theory ---------------
\begin{tikzpicture}
\node [mybox] (box){%
    \begin{minipage}{0.3\textwidth}
    \textbf{Definitions}: \\
    \vspace{-5mm}
    \setlist{nolistsep}
    \begin{itemize} [noitemsep]
        \item \textbf{Graph}: G = (V, E)
        \item \textbf{Edge}: E = E(G) is a set of edges contained in $\{\{u,v\}: u,v \in V, u \neq v\}$
        \item \textbf{Vertex}: V = V(G) is a set of vertices
        \item \textbf{Subgraph}: H = $(V_H, E_H)$ of G is a graph with $V_H \subseteq V$ and $E_H \subseteq E$
        \item \textbf{Induced Subgraph}: is a subgraph if $E_H = \{e \in E: e \subseteq V_H\}$
        \item \textbf{Component}: H of G is a maximal connected induced subgraph of G.
        \item \textbf{Degree}: $d (v) = |N (v)|$
        \item \textbf{Neighbourhood}: $N (v) = \{w \in V: \{v, w\} \in E\}$
        \item \textbf{Walk}: sequence of vertices $v_0...v_k$ such that $\{v_i, v_{i+1}\} \in E$ for all i $\le$ k-1
        \item \textbf{Length}: the value of k (see above walk definition)
        \item \textbf{Euler Walk}: a walk that contains every edge in G exactly once.
        \item \textbf{Isomorphism}: two graphs are isomorphic if there is a bijection f: $V_1 \rightarrow V_2$ such that $\{f(u), f(v)\} \in E_2$ if and only if $\{u, v\} \in E_1$
        \item \textbf{Path}: is a walk in which no vertices repeat
        \item \textbf{Connected}: A graph is connected if any two vertices are joined by a path
        \item \textbf{Digraph}: is a pair G = (V, E), V is a set of vertices and E is a set of edges contained in $\{(u, v): u, v \in V, u \neq v\}$
        \item \textbf{Strongly connected}: G is .. if for all $u, v \in V$, there is a path from u to v and a path from v to u.
        \item \textbf{Weakly connected}:
        \item \textbf{In-Neighbourhood}: $N^- (v) = \{u \in V(G): (u, v) \in E(G)\}$
        \item \textbf{Out-Neighbourhood}: $N^+ (v) = \{w \in V(G): (v, w) \in E(G)\}$
        \item \textbf{Cycle}:
        \item \textbf{Hamilton cycle}:
        \item \textbf{Bijection}:
    \end{itemize}
    \hline

    \textbf{Theorem}: If G has an Euler walk, then either: \\
    \vspace{-5mm}
    \setlist{nolistsep}
    \begin{itemize} [noitemsep]
        \item every vertex of G has even degree; or
        \item all but two vertices $v_0$ and $v_k$ have even degree, and any euler walk must have $v_0$ and $v_k$ as endpoints
    \end{itemize}
    \hline
    \end{minipage}
};
%------------ Graph Theory Header ---------------------
\node[fancytitle, right=10pt] at (box.north west) {Graph Theory};
\end{tikzpicture}









%------------ ODE Content ---------------
\begin{tikzpicture}
\node [mybox] (box){%
    \begin{minipage}{0.3\textwidth}
    \small{
    	\begin{tabular}{lp{4cm} l}
		\textit{1st Order Linear} & Use integrating factor,
        \\ & $I = e^{\int P(x) dx}$ \\ \hline
		\textit{Separable:} & $ \int P(y) dy/dx = \int Q(x) $ \\ \hline
		\textit{HomogEnEous:} & $ dy/dx = f(x,y) = f(xt,yt) $ \\ &
        sub $ y = xV $ solve, then sub $ V = y/x $ \\ \hline
        \textit{Exact:} & If $ M(x,y) + N(x,y)dy/dx = 0 $ and $ M_y = N_x $ i.e. $ \langle M,N \rangle = \nabla F $ then $ \int_x M + \int_y N = F $ \\ \hline
        \textit{Order Reduction} & Let $ v = dy/dx $ then check other types \\ 
        &\textit{If purely a function of y, }$\frac{dv}{dx} = v\frac{dv}{dy}$\\
        \hline
        \textit{Variation of Parameters:} & When $y''+a_1y'+a_2y = F(x)$ \\
        & $F$ contains $\ln x$, $\sec x$, $\tan x$, $\div$ \\ \hline
        \textit{Bernoulli} & $y' + P(x)y = Q(x)y^n$ \\
        & $\div y^n$ \\
        &$y^{-n}y'+P(x)y^{1-n}=Q(x)$ \textit{Let }$U(x) = y^{1-n}(x)$ \\
        &$\frac{dU}{dx}=(1-n)y^{-n}\frac{dy}{dx}$ \\
        &$\frac{1}{1-n}\frac{du}{dx} + P(x)U(x) = Q(x)$ \textit{solve as a 1st order} \\ \hline
        \textit{Cauchy-Euler} &$x^ny^n + a_1x^{n-1}y^{n-1} + \cdots + a_{n-1}y^{n-2}+a_ny = 0$ \\
        &guess $y = x^r$ \\
        \textit{3 Cases:} \\
        \textit{1) Distinct real roots} &$y = ax^{r_1}+bx^{r_2}$ \\
        \textit{2) Repeated real roots} &$y = Ax^r + y_2$ \\
        &\textit{Guess} $y_2 = x^ru(x)$ \\
        &\textit{Solve for $u(x)$ and choose one ($A=1, C=0$)} \\
		\textit{3) Distinct complex roots} &$y=B_1x^a \cos (b \ln x) + B_2x^a\sin (b \ln x)$
	\end{tabular}}
    \end{minipage}
};
%------------ ODE Header ---------------------
\node[fancytitle, right=10pt] at (box.north west) {ODEs};
\end{tikzpicture}

%------------ Laplace Transforms Content ---------------
\begin{tikzpicture}
\node [mybox] (box){%
    \begin{minipage}{0.3\textwidth}
    $L[f](s) = \int_0^{\infty} e^{-sx}f(x)dx $\\
    
    \small{
    	\begin{tabular}{lp{4cm} l}
        $f(t) = t^n, n \geq 0 $ &$F(s) = \frac{n!}{s^{n+1}}, s > 0 $ \\
        $f(t) = e^{at}, a \textit{ constant}$ & $ F(s) = \frac{1}{s-a}, s > a$ \\
        $f(t) = \sin{bt}, b \textit{ constant}$ & $ F(s) = \frac{b}{s^2 + b^2}, s > 0$ \\
        $f(t) = \cos{bt}, b \textit{ constant}$ & $ F(s) = \frac{s}{s^2 + b^2}, s > 0$ \\
        $f(t) = t^{-1/2}$ & $F(s) = \frac{\pi}{s^{1/2}}, s > 0$ \\
        $f(t) = \delta(t-a)$ & $F(s) = e^{-as}$ \\
        $f'$ & $L[f'] = sL[f] - f(0)$ \\
        $f''$ & $L[f''] = s^2 L[f] - sf(0) - f'(0)$ \\
        $L[e^{at}f(t)]$ & $L[f](s-a)$ \\
        $L[u_a(t)f(t-a)]$ & $L[f]e^{-as}$ 
        \end{tabular}}
    \end{minipage}
};
%------------ Laplace Transforms Header ---------------------
\node[fancytitle, right=10pt] at (box.north west) {Laplace Transforms};
\end{tikzpicture}

%------------ Vector Spaces ---------------
\begin{tikzpicture}
\node [mybox] (box){%
    \begin{minipage}{0.3\textwidth}
    $v_1, v_2 \in V$\\
    1. $v_1 + v_2 \in V$ \\
	2. $k \in \mathbb{F}, kv_1 \in V $ \\
	3. $ v_1 + v_2 = v_2 + v_1 $ \\
	4. $(v_1 + v_2) + v_3 = v_1 + (v_2 + v_3) $ \\
	5. $\forall v \in V, 0 \in V \mid 0 + v_1 = v_1 + 0 = v_1$ \\
    6. $\forall v \in V, \exists -v \in V \mid v + (-v) = (-v) + v = 0 $ \\
    7. $\forall v \in V, 1 \in \mathbb{F} \mid 1*v = v$ \\
    8. $\forall v \in V, k,l \in \mathbb{F}, (kl)v = k (lv)$ \\
    9. $\forall k \in \mathbb{F}, k(v_1 + v_2) = kv_1 + kv_2$ \\
    10. $\forall v \in V, k,l \in \mathbb{F}, (k+l)v = kv + lv$
    \end{minipage}
};
%------------ Vector Space Header ---------------------
\node[fancytitle, right=10pt] at (box.north west) {Vector Spaces};
\end{tikzpicture}
\end{multicols*}
\end{document}


Contact GitHub API Training Shop Blog About
© 2016 GitHub, Inc. Terms Privacy Security Status Help