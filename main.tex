
\documentclass{article}
\usepackage{graphicx}
\graphicspath{ {./images/} }
\usepackage[landscape]{geometry}
\usepackage{url}
\usepackage{multicol}
\usepackage{amsmath}
\usepackage{esint}
\usepackage{amsfonts}
\usepackage{tikz}
\usetikzlibrary{decorations.pathmorphing}
\usepackage{amsmath,amssymb}

\usepackage{colortbl}
\usepackage{xcolor}
\usepackage{mathtools}
\usepackage{amsmath,amssymb}
\usepackage{enumitem}
\makeatletter

\newcommand*\bigcdot{\mathpalette\bigcdot@{.5}}
\newcommand*\bigcdot@[2]{\mathbin{\vcenter{\hbox{\scalebox{#2}{$\m@th#1\bullet$}}}}}
\makeatother

\title{Algorithms II CheatSheet}
\usepackage[brazilian]{babel}
\usepackage[utf8]{inputenc}

\advance\topmargin-.8in
\advance\textheight3in
\advance\textwidth3in
\advance\oddsidemargin-1.5in
\advance\evensidemargin-1.5in
\parindent0pt
\parskip2pt
\newcommand{\hr}{\centerline{\rule{3.5in}{1pt}}}
%\colorbox[HTML]{e4e4e4}{\makebox[\textwidth-2\fboxsep][l]{texto}
\begin{document}

\begin{center}{\huge{\textbf{Algorithms II Cheat-Sheet}}}\\
\end{center}
\begin{multicols*}{3}

\tikzstyle{mybox} = [draw=black, fill=white, very thick,
    rectangle, rounded corners, inner sep=10pt, inner ysep=10pt]
\tikzstyle{fancytitle} =[fill=black, text=white, font=\bfseries]

%------------ Notation ---------------
\begin{tikzpicture}
\node [mybox] (box){%
    \begin{minipage}{0.3\textwidth}
    $A \in [10] \equiv A \in [1..10]$ \\
    \{a, b c\} is a set of vertices \\
    G\{a, b, c\} is a graph \\
    $\overrightarrow{x}$ just represents a vector $x$
    \end{minipage}
};
%------------ Notation Header ---------------------
\node[fancytitle, right=10pt] at (box.north west) {Notation};
\end{tikzpicture}
    
%------------ Big O ---------------
\begin{tikzpicture}
\node [mybox] (box){%
    \begin{minipage}{0.3\textwidth}
        \includegraphics[width=8cm]{big-o-definition.png}
        \includegraphics[width=8cm]{big-o-formal.png}
    \end{minipage}
};
%------------ Big O Header ---------------------
\node[fancytitle, right=10pt] at (box.north west) {Big O};
\end{tikzpicture}

%------------ Interval Scheduling ---------------
\begin{tikzpicture}
\node [mybox] (box){%
    \begin{minipage}{0.3\textwidth}
    \includegraphics[width = 8cm]{req-comp-def.png}
    \hline
    \includegraphics[width = 8cm]{interval-scheduling-problem.png}
    \hline
    \includegraphics[width=8cm]{greedy-schedule-pseudo.png}
    \textbf{Complexity:} \\
    Step 2 takes O(n log n) \\
    Steps 3–6 all take O(1) time and are executed at most n times. \\
    $\therefore total running time = O(n log n) + O(n) · O(1) = O(n log n).$
    \end{minipage}
};
%------------ Interval Scheduling Header ---------------------
\node[fancytitle, right=10pt] at (box.north west) {Interval Scheduling};
\end{tikzpicture}


%------------ Interval Scheduling 2 ---------------
\begin{tikzpicture}
\node [mybox] (box){%
    \begin{minipage}{0.3\textwidth}
    
    \textbf{Formal GreedySchedule}: \\
    $A^+$ := argmin $\{f : (s, f ) \in R, A \cup \{(s, f )\}$ is compatible\} for all $A \subseteq R$, \\
    $A_0 := \emptyset,$ \hspace{1cm} $A_{i+1} := A_i \cup \{A_i^+\}$ \\
    t := max\{i: $A_i$ is defined\}\\
    \textbf{Interval Scheduling Proofs} \\
    \textbf{Lemma}: Greedy Schedule always outupts $A_t$ \\
    \textbf{Proof}: By induction form the following loop invariant.
    At the start of the i'th iteration of 4-7: \\
    \vspace{-5mm}
    \setlist{nolistsep}
    \begin{itemize} [noitemsep]
        \item A is equal to $A_t \cap \{(s_1, f1), ..., (s_{i-1}, f_{i-1})\}$
        \item lastf is equal to the latest finish time of any request in A (or 0 if A = [])
    \end{itemize}
    \hline
    \textbf{Lemma}: $A_t$ is a compatible set \\
    \textbf{Proof}: Instant by induction; $A_0$ is compatible, and if $A_i$ is compatible then so is $A_{i+1} = A_i \cup {A^+}$ by the definition of $A_i^+$
    
    \hline
    \textbf{Lemma}: $A_t$ is a maximum compatible subset of the Array $R$ (look in pseudocode) \\
    \textbf{Proof}:\\
    Base case for i = 1: $A_0^+$ is the fastest finishing request in $R$ by definition\\
    Inductive step: Suppose $A_i$ finishes faster than $B_i$. \\
    Let $B_i^+$ be the (i+1)'st fastetst-finishing element of B.
    Since $A_i$ finishes faster than $B_i$, $A_i \cup \{B_i^+\}$ is compatible. Hence by definition, $A_i^+$ exists and finishes no later than $B_i^+$
    \hline
    \textbf{Theorem}: GreedySchedule outputs $A_t$, which is a maximum compatible set. \\
    \textbf{Proof}: putting all of the above proofs together, we prove the theorem.
    \end{minipage}
};
%------------ Interval Scheduling 2 Header ---------------------
\node[fancytitle, right=10pt] at (box.north west) {Interval Scheduling};
\end{tikzpicture}

%------------ Graph Theory ---------------
\begin{tikzpicture}
\node [mybox] (box){%
    \begin{minipage}{0.3\textwidth}
    
    \textbf{Graph}: G = (V, E) \\
    \textbf{Edge}: E = E(G) is a set of edges contained in $\{\{u,v\}: u,v \in V, u \neq v\}$ \\
    \textbf{Vertex}: V = V(G) is a set of vertices \\
    \textbf{Subgraph}: H = $(V_H, E_H)$ of G is a graph with $V_H \subseteq V$ and $E_H \subseteq E$ \\
    \textbf{Induced Subgraph}: is a subgraph if $E_H = \{e \in E: e \subseteq V_H\}$ \\
    \textbf{Component}: H of G is a maximal connected induced subgraph of G. \\
    \textbf{Degree}: $d (v) = |N (v)|$ \\
    \textbf{Neighbourhood}: $N (v) = \{w \in V: \{v, w\} \in E\}$ \\
    \textbf{Walk}: sequence of vertices $v_0...v_k$ such that $\{v_i, v_{i+1}\} \in E$ for all i $\le$ k-1 \\
    \textbf{Length}: the value of k (see above walk definition) \\
    \textbf{Euler Walk}: a walk that contains every edge in G exactly once. \\
    \textbf{Isomorphism}: two graphs are isomorphic if there is a bijection f: 
    $V_1 \rightarrow V_2$ such that $\{f(u), f(v)\} \in E_2$ if and only if $\{u, v\} \in E_1$ \\
    \textbf{Path}: is a walk in which no vertices repeat \\
    \textbf{Connected}: A graph is connected if any two vertices are joined by a path \\
    \textbf{Digraph}: is a pair G = (V, E), V is a set of vertices and E is a set of edges contained in $\{(u, v): u, v \in V, u \neq v\}$ \\
    \textbf{Strongly connected}: G is .. if for all $u, v \in V$, there is a path from u to v and a path from v to u. \\
    \textbf{Weakly connected}: \\
    \textbf{In-Neighbourhood}: $N^- (v) = \{u \in V(G): (u, v) \in E(G)\}$ \\
    \textbf{Out-Neighbourhood}: $N^+ (v) = \{w \in V(G): (v, w) \in E(G)\}$ \\
    \textbf{Cycle}: is a walk $W = w_0...w_k$ with $ w_0 = w_k$ and $k \ge 3$,
    in which every vertex appears at most once except for $w_0$ and $w_k$ (which appear twice) \\
    \textbf{Hamilton cycle}: is a cycle containing every vertex in the graph \\
    \textbf{k-regular}: a graph is .. if every vertex has degree k\\
    \textbf{Bijection}: \\
    \end{minipage}
};
%------------ Graph Theory Header ---------------------
\node[fancytitle, right=10pt] at (box.north west) {Graph Theory};
\end{tikzpicture}



%------------ Graph Theory 2 ---------------
\begin{tikzpicture}
\node [mybox] (box){%
    \begin{minipage}{0.3\textwidth}

    \textbf{Theorem}: If G has an Euler walk, then either: \\
    \vspace{-5mm}
    \setlist{nolistsep}
    \begin{itemize} [noitemsep]
        \item every vertex of G has even degree; or
        \item all but two vertices $v_0$ and $v_k$ have even degree, and any euler walk must have $v_0$ and $v_k$ as endpoints
    \end{itemize}
    \hline
    \textbf{Theorem}: let G = (V, E) be a digraph with no isolated vertices, and let $U,v \in V$. Then G has an Euler walk 
    from u to v if and only if G is weakly connected and either: \\
    \vspace{-5mm}
    \setlist{nolistsep}
    \begin{itemize} [noitemsep]
        \item u = v and every vertex of G has equal in- and out-degrees; order
        \item $u \neq v, d^+ (u) = d^- (u) + 1, d^- (v) = d^+ (v) + 1$ and every other vertex of G has equal in- and out-degrees
    \end{itemize}
    \hline
    \textbf{Dirac's Theorem}: Let $n \ge 3$. Then any n-vertex graph G with minimum degree at least $\frac{n}{2}$ has a Hamilton cycle. \\
    % \textbf{Proof}: Try to find a long path inductively: start with a trivial one-vertex path and repeatedly extend it. \\ TODO PROOF
    % Suppose G contains a k-vertex path v1 . . . vk for some k ∈ [n - 1]. \\
    % \textbf{Case 1: $k \le \frac{n}{2}$}: then being greedy works. \\
    % In general, $d(v_k) \ge \frac{n}{2} $>$ |\{v_1, ..., v_{k-1}\}|$, so there’s a vertex $v{k+1}$ adjacent to $v_k$ other than $v_1, . . . , v_{k-1}$. 
    % Then $v_1 . . . v_{k+1}$ is a path of length k + 1. 

    \textbf{Handshake lemma}: For any graph \\G = (V, E), $\sum_{v \in V}^{} d(v) = 2|E|$ \\
    \textbf{Proof}: All edges contain two vertices, and eachvertex v is in d(v) edges. Count the number of verted-edge pairs:
    Let $X = \{(v, e) \in V \times E: v \in E\}$. Then $|X| = 2|E|$ and $|X|$ = $\sum_{v \in V}^{}d(v)$, so we're done.
    \hline

    \textbf{Directed Handshake lemma}: For any graph \\G = (V, E), $\sum_{v \in V}^{} d^+(v) = \sum_{v \in V}^{} d^-(v) = 2|E|$ \\
    \textbf{Proof}: TODO. Instead of counting vertex-edge pairs, we count tail-edge pairs. Each edge has one tail so $|X| = |E|$
    \hline


    \end{minipage}
};
%------------ Graph Theory 2 Header ---------------------
\node[fancytitle, right=10pt] at (box.north west) {Graph Theory};
\end{tikzpicture}


%------------ Trees ---------------
\begin{tikzpicture}
\node [mybox] (box){%
    \begin{minipage}{0.3\textwidth}
    
    \textbf{Forest}: a graph with no cycles \\
    \textbf{Tree}: a forest that is connected \\
    \textbf{Root}: for T = (V, E). Root $r \in V$ as follows. For all vertices $v \neq r$, let $P_v$ be the unique path from r to v.
    Then direct each $P_v$ from r to v. \\
    \textbf{Leaf}: is a degree-1 vertex. Root cannot be a leaf. \\
    \textbf{Ancestor}: u is an .. of v if u i on $P_v$ \\
    \textbf{Parent}: u is the .. of v if $u \in N^-(v)$ \\
    \textbf{level}: first .. $L_0$ of T is {r}, and $L_{i+1} = N^+(L_i)$. \\
    \textbf{depth}: of T is max\{i: $L_i \neq \emptyset$\}. Root doesn't count  \\
    \hline
    \textbf{Lemma 1}: If T = (V, E) is a tree, then any pair of vertices u, v $\in$ V is joined by a unique path uTv in T. \\
    % \textbf{Proof}: TODO \\
    \textbf{Lemma 2}: Any n-vertex tree has n-1 edges \\
    % \textbf{Proof}: TODO \\
    \textbf{Lemma 3}: Any n-vertex tree T = (V, E) with $n \ge 2$ has at least 2 leaves \\
    % \textbf{Proof}: TODO \\
    \hline
    \textbf{Tree Properties}: \\
    \textbf{A}: T is connected and has no cycles \\
    \textbf{B}: T has n-1 edges and is connected \\
    \textbf{C}: T has n-1 edges and has no cycles \\
    \textbf{D}: T has a unique path between any pair of vertices \\
    $A \implies B, C, D \hspace{2cm} A \impliedby B, C, D$. \\
\end{minipage}
};
%------------ Trees Header ---------------------
\node[fancytitle, right=10pt] at (box.north west) {Trees};
\end{tikzpicture}



%------------ Depth First Search ---------------
\begin{tikzpicture}
\node [mybox] (box){%
    \begin{minipage}{0.3\textwidth}
    \textbf{Graphs as data structures}: \\
    \textbf{Adjacency Matrix}: \\
    Storing: $\Theta(|V|^2)$ space \\
    Adjacency query: $\Theta(1)$ time \\
    Neighbourhood query: $\Theta(|V|)$ time \\
    \textbf{Adjacency List}: \\
    \includegraphics[width = 3cm]{adjacency-list.png} \\
    Storing: $\Theta(|V| + |E|)$ space \\
    Adjacency query: $\Theta(d^+(u))$ time \\
    Neighbourhood query: $\Theta(d^+(u))$ time
    \hline
    \textbf{DFS}: \\
    \includegraphics[width = 8cm]{dfs-pseudocode.png}
    \textbf{Complexity}: In total there are $\sum_{v \in V}^{}d(v) = O(|E|)$ calls to helper (each vertex only runs lines 5-7 once), 
    and there is O(1) time between calls. So the running time is $O(|V| + |E|)$.\\
    \textbf{Invariant}: When helper is called, if explored[i] = 1 then $v_i \in V(C)$. \\
    \textbf{Claim}: Every vertex in P is explored \\
    \textbf{Proof by induction}: We prove $x_1, ..., x_i$ are explored for all $i \le t$.
    $x_1$ is explored. If $x_i$ is explored, then helper($x_{i+1}$) will be called from helper($x_i$). so $x_{i+1}$ will also be explored. \\
    \textbf{DFS Tree}: a .. T of G is a rooted tree satisfying: \\
    \vspace{-5mm}
    \setlist{nolistsep}
    \begin{itemize} [noitemsep]
        \item $V(T)$ is the vertex set of a component of G;
        \item If $\{x, y\} \in E(G)$, then x is an ancestor of y in T or vice versa.
    \end{itemize}
    
    \end{minipage}
};
%------------ Depth First Search Header ---------------------
\node[fancytitle, right=10pt] at (box.north west) {Depth First Search};
\end{tikzpicture}

%------------ Breadth First Search ---------------
\begin{tikzpicture}
\node [mybox] (box){%
    \begin{minipage}{0.3\textwidth}
    
        \textbf{Distance}: The distance between x and y, d(x, y), is the length in edges of a
        shortest path between x and y, or $\infty$ if no such path exists. \\
        \textbf{BFS}: \\
        \includegraphics[width = 8cm]{bfs-pseudocode.png}
        \textbf{Complexity}: If G is in adjacency list form, each edge is added to queue at most twice,
        incurring $O(1)$ overhead each time, so the running time is $O(|V| + |E|)$. \\
        \includegraphics[width = 6cm]{bfs-graph.png}
        \textbf{BFS explanation}: BFS works by starting at a vertex and then adding all adjacent vertices to a queue.
        We then take the first vertex in the queue and look for a new set of adjacent vertices to add,
        repeating the process until we have reached our destination. \\


    \end{minipage}
};
%------------ Breadth First Search Header ---------------------
\node[fancytitle, right=10pt] at (box.north west) {Breadth First Search};
\end{tikzpicture}



%------------ Dijkstra's Algorithm ---------------
\begin{tikzpicture}
\node [mybox] (box){%
    \begin{minipage}{0.3\textwidth}
        \textbf{Weighted Graph}: is a pair (G, w), where G is a graph and $w: E(G) \rightarrow \mathbb{R}$ is a \textbf{weight function}\\
        \textbf{Length}: of a path/walk $P = x_1 \dots x_t$ is the total weight of P's edges: 
        length(P) = $\sum_{i=1}^{t-1}w(x_i, x_{i+1})$. \\
        \textbf{Distance}: from x to y is the shortest length of any path/walk from x to y, or $\infty$ if they are in different components \\
        \textbf{Priority queue}: each element has a priority, and the first element is the one with the lowest priority. \\
        \textbf{Dijkstra}: \\
        \includegraphics[width = 8cm]{dijkstra-pseudocode.png}
        \textbf{Complexity}: We perform $O(|V|)$ Insert operations and Extract operations, and $O(|E|)$ ChangeKey
        operations, for a total of $O((|V| + |E|) log |V|)$ time when G is given in adjacency list form. \\
        \textbf{Dijkstra Operations}: \\
        \vspace{-5mm}
        \setlist{nolistsep}
        \begin{itemize} [noitemsep]
            \item StartQueue(n) returns a new priority queue of maximum length n.
            \item Insert(x, p) inserts a new element x with priority p.
            \item Exctract() removes and returns the lowest-priority element.
            \item ChangeKey(x, p) udpates the priority of x to p.
            \item StartQueue takes $O(n)$ time, all other operations take $O(log(n))$ time.
        \end{itemize}
    \end{minipage}
};
%------------ Dijkstra's Algorithm Header ---------------------
\node[fancytitle, right=10pt] at (box.north west) {Dijkstra's Algorithm};
\end{tikzpicture}




%------------ Linear Programming ---------------
\begin{tikzpicture}
\node [mybox] (box){%
\begin{minipage}{0.3\textwidth}
    
    % \textbf{Linear Programming}: TODO\\
    \textbf{Feasible}: We say $\overrightarrow{x} \in \mathbb{R}^n$ is a feasible solution if $\overrightarrow{x} \ge \overrightarrow{0}$ 
    and $ A\overrightarrow{x} \le \overrightarrow{b}.$\\
    \textbf{Optimal}: We say $\overrightarrow{x}$ is an optimal solution if $f(\overrightarrow{y}) \le f(\overrightarrow{x})$ for all
    feasible $y \in \mathbb{R}^n$\\
    \textbf{Polytope}: is a geometric object with flat sides \\
    \textbf{Corollary}: There will always be an optimal solution \\

    \textbf{Non-Standard Form}: \\
    $-4x + 5y - z \rightarrow$ max subject to \\
    $x + y + z \le 5$; \\
    $x + y + z \ge 5$; \\
    $x + 2y \ge 2$; \\
    $x, z \ge 0$.

    \textbf{Standard Form}: \\
    $-4x + 5(y_1 - y_2) - z \rightarrow$ max subject to \\
    $x + (y_1 - y_2) + z \le 5;$ \\
    $-x - (y_1 - y_2) - z \le -5;$ \\
    $-x - 2(y_1 - y_2) \le -2;$ \\
    $x, y_1, y_2, z \ge 0.$ \\

    \textbf{Matrix Form}: \\
    \includegraphics[width = 5cm]{linear-program-matrix-form.png}

    \textbf{Simplex Method}: Search greedily for a vertex of the feasible polytope which maximises the objective function \\
    Worst case: hypercube which has $\Omega(2^n)$ vertices. \\
    In practice it only need $\Theta(n)$ steps \\
    \textbf{Vertex Cover}: in a graph G, is a set $X \subseteq V$ such that every edge in E has at least one vertex in X\\
    We can express finding a minimum vertex cover as solving a linear program in which the solutions must be integers: 
    an integer linear program. \\
    \includegraphics[width = 8cm]{vertex-cover-example.png}
    \end{minipage}
};
%------------ Linear Programming Header ---------------------
\node[fancytitle, right=10pt] at (box.north west) {Linear Programming};
\end{tikzpicture}

%------------ Flow Networks ---------------
\begin{tikzpicture}
\node [mybox] (box){%
    \begin{minipage}{0.3\textwidth}
    
        
        \textbf{Flow Network}: consists of a directed graph $G = (V, E)$,
        a \textbf{capacity function} $c: E \rightarrow \mathbb{N}$,
        a \textbf{source} vertex $s \in V$ with $N^-(s) = \emptyset$,
        and a \textbf{sink} vertex $t \in V$ with $N^+(t) = \emptyset$ \\
        \textbf{Flow}: is a function in (G, c, s, t) $f: E \rightarrow \mathbb{R}$
        with properties:
        % \vspace{-5mm}
        \setlist{nolistsep}
        \begin{itemize} [noitemsep]
            \item No edge has more flow than capacity; \\formally, for all $e \in E, 0 \le f(e) \le c(e)$
            \item Flow is conserved at vertices; flow in = flow out
        \end{itemize}
        \textbf{Maximum Flow}: a flow $f$ maximising the value of the flow, $v(f)$\\
        \textbf{Cut}: is any pair of disjoint edges $A, B \subseteq V$
        with $A \cup B = V$, $s \in A$ and $t \in B$. \\
        \hline
        
        \textbf{Lemma 1}: For all sets $X \subseteq V \ \{s, t\}$, we have $f^+(X) = f^-(X)$.
        So flow is conserved in sets/cuts as well as vertices \\
        \textbf{Proof}: By summing conservation of flow over all $v \in X$: \\
        $\sum_{v \in X}^{} \sum_{u \in N^-(v)}^{} f(u, v) = \sum_{v \in X}^{} \sum_{w \in N^+(v)}^{} f(v, w)$.
        For all $e \subseteq X$, $f(e)$ appears once on each side; after cancelling those
        terms we're left with $f^+(X) = f^-(X)$. \\
        \textbf{Lemma 2}: For all cuts $(A, B)$, $f^+(A) - f^-(A) = f^-(B) - f^+(B)$. \\
        \textbf{Proof}: We have shown that $v(f) = f^+(A) - f^-(A)$ because A and B are disjoint and $A \cup B = V.$
        \hline
        \textbf{Lemma 3}: Push(G, c,s,t, f , P) returns a new flow $f'$, with value $v(f') = v(f) + C$ in $O(|V(G)|)$ time\\
        \textbf{Ford-Fulkerson}: \\
        \includegraphics[width = 8cm]{ford-fulkerson-pseudocode.png}
        \textbf{Complexity}: Every step takes $O(|E|)$ time or $O(|V|)$ time, and since G is weakly connected we have 
        $|V| = O(|E|)$. So the running time is $O(v(f*)|E|)$.\\
        
        
    
    \end{minipage}
};
%------------ Flow Networks Header ---------------------
\node[fancytitle, right=10pt] at (box.north west) {Flow Networks};
\end{tikzpicture}




%------------ Flow Networks 2 ---------------
\begin{tikzpicture}
    \node [mybox] (box){%
        \begin{minipage}{0.3\textwidth}
            \textbf{Residual graph}: $G_f$ of (G, c, s, t) on V(G) as follows:\\
            % \vspace{-5mm}
            \setlist{nolistsep}
            \begin{itemize} [noitemsep]
                \item if flow $<$ capacity: then forward edge with value capacity-flow
                \item if flow $>$ 0: add backward edge with value flow
            \end{itemize}
            \includegraphics[width = 8cm]{residual-graph-example.png}
            \textbf{Residual capacity of edge}: max\{capacity - flow, backward edge flow\}\\
            \textbf{Residual capacity of network}: minimum residual capacity of it's edges\\
            \textbf{Augmenting Path}: \\
            \textbf{Max-flow min-cut theorem}: The value of a maximum flow is equal to the minimum capacity of a cut, i.e. the minimum
            value of $c^+(A)$ over all cuts (A, B). \\
            \textbf{Proof}: Let f be a maximum flow, and let (A, B) be a cut minimising
            $c^+(A)$. We already proved $v(f) \le c^+(A)$. Moreover, there is no
            augmenting path for f , so exactly as before, there is a cut $(A'. B')$ with
            $c^+(A') = v(f)$; thus $v(f) \ge c^+(A)$. The result follows. \\
            
        
        \end{minipage}
    };
    %------------ Flow Networks 2 Header ---------------------
    \node[fancytitle, right=10pt] at (box.north west) {Flow Networks};
    \end{tikzpicture}



\end{multicols*}
\end{document}